\documentclass[11pt,twocolumn]{article}

% ─── Packages ──────────────────────────────────────────────
\usepackage[utf8]{inputenc}
\usepackage[T1]{fontenc}
\usepackage{amsmath,amssymb,amsthm}
\usepackage{mathtools}
\usepackage{bm}
\usepackage{graphicx}
\graphicspath{{figures/}{./}}
\usepackage{booktabs}
\usepackage{algorithm}
\usepackage{algorithmic}
\usepackage[margin=1in]{geometry}
\usepackage{hyperref}
\usepackage{cleveref}
\usepackage{natbib}
\usepackage{xcolor}
\usepackage{thmtools}
\usepackage{caption}

% ─── Theorem environments ──────────────────────────────────
\newtheorem{theorem}{Theorem}[section]
\newtheorem{lemma}[theorem]{Lemma}
\newtheorem{proposition}[theorem]{Proposition}
\newtheorem{corollary}[theorem]{Corollary}
\newtheorem{definition}[theorem]{Definition}
\newtheorem{remark}[theorem]{Remark}

% ─── Custom commands ───────────────────────────────────────
\newcommand{\R}{\mathbb{R}}
\newcommand{\norm}[1]{\left\lVert #1 \right\rVert}
\newcommand{\abs}[1]{\left| #1 \right|}
\newcommand{\eisv}{\mathbf{x}}
\newcommand{\deta}{\Delta\bm{\eta}}
\newcommand{\cv}{C(V,\Theta)}
\newcommand{\todo}[1]{\textcolor{red}{\textbf{[TODO: #1]}}}

% ─── Metadata ──────────────────────────────────────────────
\title{UNITARES: Contraction-Theoretic Governance of\\
Autonomous Agent Thermodynamics}

\author{
  Kenny Wang\\
  Independent Researcher\\
  \texttt{founder@cirwel.org}
}

\date{Draft --- \today}

% ═══════════════════════════════════════════════════════════
\begin{document}
\maketitle

% ─── Abstract ──────────────────────────────────────────────
\begin{abstract}
We present UNITARES, a thermodynamically-inspired governance framework for
autonomous AI agents based on four coupled state variables:
Energy~($E$), Information integrity~($I$), Entropy~($S$), and
Void~($V$). The system's dynamics are governed by ordinary differential
equations whose stability we establish through contraction theory on a
Riemannian state manifold. We prove global exponential convergence to a
unique equilibrium under a metric $M \succ 0$ and derive conditions for
contraction rate $\alpha > 0$ as a function of the system parameters.

Three novel contributions extend the theoretical core.
First, we identify and resolve a boundary saturation pathology in the
information integrity channel, providing both a diagnostic criterion
and a corrected dynamics mode.
Second, we introduce an adaptive PID governor (CIRS~v2) that treats
governance thresholds as per-agent state variables with phase-aware
reference tracking, replacing static configuration with principled
control.
Third, we define a concrete ethical drift vector $\deta \in [0,1]^4$
composed of four observable signals---calibration deviation, complexity
divergence, coherence deviation, and stability deviation---grounding
what has been an abstract concept in measurable engineering quantities.

We validate the framework on a production deployment of 903~agents
over 69~days, reporting equilibrium statistics, regime distributions,
and saturation analysis from 198{,}333 audit events. The observed
EISV equilibrium ($\bar{E} = 0.77$, $\bar{I} = 0.88$, $\bar{S} =
0.08$, $\bar{V} = -0.03$) confirms theoretical predictions, with
100\% of active agents maintaining $V$ within the predicted
$[-0.1, 0.1]$ operating range. All 75 active agents exhibit
$m_{\mathrm{sat}} < 0$ under logistic information dynamics, confirming
universal I-channel saturation and motivating the linear correction.
\end{abstract}

\textbf{Keywords:} autonomous governance, contraction theory,
thermodynamic dynamics, adaptive control, ethical drift, AI safety

% ═══════════════════════════════════════════════════════════
\section{Introduction}
\label{sec:intro}

Autonomous AI agents operating in production environments face a
fundamental tension: they must explore to learn, yet remain safe while
doing so. Existing approaches to agent governance fall into two camps.
\emph{Hard constraints} (guardrails, output filters) provide safety
but impede capability. \emph{Soft monitoring} (logging, dashboards)
preserves capability but offers no formal guarantees.

We argue that governance should be \emph{intrinsic}---a continuous
dynamical process that shapes agent behavior from within, much as
homeostatic regulation governs biological organisms. This paper
presents UNITARES (\textbf{Uni}fied \textbf{T}hermodynamic
\textbf{A}daptive \textbf{Re}gulation \textbf{S}ystem), a framework
in which each agent carries a four-dimensional thermodynamic state
$\eisv = (E, I, S, V)^\top$ whose evolution is governed by coupled
differential equations. The state provides a continuous, interpretable
signal for governance decisions---when to proceed, when to pause, when
to escalate.

\subsection{Contributions}

This paper makes six contributions:

\begin{enumerate}
\item \textbf{EISV Dynamics} (\Cref{sec:dynamics}): A coupled ODE
  system on $[0,1]^2 \times [0,2] \times [-2,2]$ with coherence
  feedback, ethical drift coupling, and entropy-driven uncertainty
  tracking.

\item \textbf{Contraction Analysis} (\Cref{sec:contraction}): We prove
  global exponential stability under a Riemannian metric $M \succ 0$
  and derive the contraction rate $\alpha$ as a function of system
  parameters, providing concrete conditions for safe operation.

\item \textbf{I-Channel Saturation Resolution} (\Cref{sec:saturation}):
  We identify a boundary saturation pathology in the logistic
  information dynamics, derive a diagnostic criterion, and present a
  corrected linear mode that prevents boundary accumulation.

\item \textbf{Adaptive Governor} (\Cref{sec:governor}): A PID-based
  controller that manages governance thresholds as living per-agent
  state, with phase-aware reference tracking, integral wind-up
  protection, and multi-agent coordination.

\item \textbf{Concrete Ethical Drift} (\Cref{sec:ethical-drift}): A
  four-component vector $\deta \in [0,1]^4$ measured from observable
  signals, replacing abstract ethics with engineering quantities.

\item \textbf{Production Validation} (\Cref{sec:experiments}):
  Empirical results from a deployment of 903 agents over 69~days,
  including EISV equilibrium statistics, regime distributions,
  I-channel saturation analysis, and knowledge graph metrics.
\end{enumerate}

\subsection{Related Work}

\paragraph{Contraction Theory.}
Lohmiller and Slotine~\cite{lohmiller1998contraction} established
contraction analysis as a tool for nonlinear stability, later extended
to Riemannian metrics~\cite{slotine2003modular}. We adapt this
framework to the governance domain, where the ``plant'' is an agent's
thermodynamic state rather than a physical system.

\paragraph{AI Safety and Governance.}
Constitutional AI~\cite{bai2022constitutional} and
RLHF~\cite{ouyang2022training} govern model outputs through training
objectives, while output filters and guardrails~\cite{rebedea2023nemo}
enforce constraints at inference time. These approaches treat safety as
a discrete property (safe/unsafe) applied at the boundary of agent
action. UNITARES instead models safety as a continuous dynamical
state that evolves between actions, providing leading indicators of
degradation rather than binary verdicts after the fact.

Ravindran~\cite{ravindran2025moral} proposes a Moral Anchor System
(MAS) that detects value drift using a three-dimensional state vector
with Bayesian inference and LSTM-based forecasting. While MAS and
UNITARES share the goal of detecting ethical drift, they differ
fundamentally: MAS is a statistical detection pipeline that predicts
drift from observed behavior, whereas UNITARES models drift as a
dynamical force that couples into the state equations, providing
contraction-theoretic stability guarantees that statistical methods
cannot offer.

Control Barrier Functions (CBFs)~\cite{ames2019control} enforce
forward invariance of safe sets for multi-agent systems. CBFs
guarantee that the system remains within a safe region but do not
provide convergence to a desired operating point. Contraction theory,
by contrast, guarantees exponential convergence to a unique
trajectory, a strictly stronger property that subsumes set invariance
when the equilibrium lies within the safe set.

\paragraph{Adaptive Control.}
PID control is classical~\cite{astrom2006advanced}; our contribution
is applying it to governance thresholds with phase-aware reference
tracking, connecting it to the thermodynamic state through the
coherence function.

\paragraph{Calibration.}
Confidence calibration has a rich literature in forecasting and ML
\cite{guo2017calibration,niculescu2005predicting}. Our ethical drift
vector uses calibration deviation as one of four measurable components,
connecting calibration to governance dynamics.

% ═══════════════════════════════════════════════════════════
\section{EISV Thermodynamic Dynamics}
\label{sec:dynamics}

\subsection{State Space}

The UNITARES state is a four-dimensional vector
$\eisv = (E, I, S, V)^\top$ evolving on the compact set
\[
  \mathcal{X} = [0,1] \times [0,1] \times [0, S_{\max}] \times [-V_{\max}, V_{\max}]
\]
with $S_{\max} = 2$ and $V_{\max} = 2$. Each component has a
physical interpretation:

\begin{definition}[EISV State Variables]
\label{def:eisv}
\begin{itemize}
\item $E \in [0,1]$: \textbf{Energy}---the agent's productive
  capacity, analogous to thermodynamic energy.
\item $I \in [0,1]$: \textbf{Information integrity}---the reliability
  and consistency of the agent's knowledge and outputs.
\item $S \in [0, S_{\max}]$: \textbf{Entropy}---semantic uncertainty,
  bounded below by an epistemic humility floor $S_{\min} = 0.001$.
\item $V \in [-V_{\max}, V_{\max}]$: \textbf{Void}---the $E$-$I$
  imbalance integral, a damped accumulator analogous to Helmholtz free
  energy. $V > 0$ indicates energy surplus; $V < 0$ indicates
  integrity surplus.
\end{itemize}
\end{definition}

\subsection{Governing Equations}
\label{sec:equations}

The dynamics are:
\begin{align}
\frac{dE}{dt} &= \alpha(I - E) - \beta_E E S + \gamma_E \norm{\deta}^2
  \label{eq:dE} \\
\frac{dI}{dt} &= -kS + \beta_I \cv - g_I(I)
  \label{eq:dI} \\
\frac{dS}{dt} &= -\mu S + \lambda_1(\Theta)\norm{\deta}^2
  - \lambda_2(\Theta)\cv \nonumber \\
  &\quad + \beta_c \mathcal{C} + \sigma\xi(t)
  \label{eq:dS} \\
\frac{dV}{dt} &= \kappa(E - I) - \delta V
  \label{eq:dV}
\end{align}

where $g_I(I)$ is the information integrity self-regulation term
(\Cref{sec:saturation}), $\cv$ is the coherence function
(\Cref{sec:coherence}), $\norm{\deta}$ is the ethical drift norm
(\Cref{sec:ethical-drift}), $\mathcal{C} \in [0,1]$ is task
complexity, and $\xi(t)$ is a noise process with intensity $\sigma$.

\begin{remark}
The coupling structure encodes physical intuition:
$E$ tracks $I$ (energy follows integrity) while being eroded by entropy;
$I$ is degraded by uncertainty but restored by coherence;
$S$ decreases naturally but is driven up by ethical drift and complexity;
$V$ integrates the $E$-$I$ gap with damping $\delta$.
\end{remark}

\subsection{Coherence Function}
\label{sec:coherence}

The coherence function provides a smooth, bounded feedback from the
void integral to the information channel:
\begin{equation}
  \cv = C_{\max} \cdot \tfrac{1}{2}\left(1 + \tanh(C_1 \cdot V)\right)
  \label{eq:coherence}
\end{equation}
with $C_{\max} = 1$ and $C_1 = 1.0$ (production values). This maps
$V \in \R \to [0, C_{\max}]$ with the steepest sensitivity at $V = 0$.

\begin{remark}[Operating Range]
\label{rem:operating-range}
In production, the damping term $\delta V$ in \eqref{eq:dV} constrains
$V$ to approximately $[-0.1, 0.1]$, yielding $C(V) \approx 0.49$. This
narrow operating range is not a deficiency but an honest signal: the
system operates conservatively with integrity consistently exceeding
energy ($I > E$).
\end{remark}

\subsection{Adaptive Entropy Coupling}

The entropy source term $\lambda_1(\Theta)$ is adaptive, coupling the
governance parameter vector $\Theta$ to the dynamics:
\begin{equation}
  \lambda_1(\Theta) = \lambda_{1,\min} +
  \frac{\eta_1 - \eta_{1,\min}}{\eta_{1,\max} - \eta_{1,\min}}
  \cdot (\lambda_{1,\max} - \lambda_{1,\min})
  \label{eq:lambda1}
\end{equation}
where $\eta_1 \in [0.1, 0.5]$ maps to $\lambda_1 \in [0.05, 0.20]$.
Higher ethical drift sensitivity increases entropy injection, creating
a natural feedback loop: drifting agents become more uncertain, which
in turn triggers conservative governance decisions.

\subsection{Objective Function}

The scalar governance signal is:
\begin{equation}
  \Phi = w_E E - w_I(1-I) - w_S S - w_V\abs{V} - w_\eta\norm{\deta}^2
  \label{eq:phi}
\end{equation}
with default weights $w_E = w_I = w_S = w_V = w_\eta = 0.5$.
Governance verdicts follow thresholds: $\Phi \geq 0.15$ (proceed),
$0 \leq \Phi < 0.15$ (caution), $\Phi < 0$ (high-risk/pause).

\begin{remark}
The threshold $\Phi = 0.15$ was calibrated empirically: a healthy agent
with $E = 0.7$, $I = 0.8$, $S = 0.2$ yields $\Phi \approx 0.15$.
Earlier thresholds ($\Phi = 0.3$) were too aggressive, triggering
unnecessary caution for well-functioning agents.
\end{remark}


% ═══════════════════════════════════════════════════════════
\section{Contraction Analysis}
\label{sec:contraction}

We establish stability of the EISV system using contraction theory
on a Riemannian manifold.

\subsection{Contraction Framework}

\begin{definition}[Contraction~\cite{lohmiller1998contraction}]
A system $\dot{\eisv} = f(\eisv, t)$ is \emph{contracting} with rate
$\alpha > 0$ in a metric $M(\eisv, t) \succ 0$ if the generalized
Jacobian satisfies
\begin{equation}
  F = \left(\dot{M} + M\frac{\partial f}{\partial \eisv}
  + \left(\frac{\partial f}{\partial \eisv}\right)^\top M\right)
  \preceq -2\alpha M
  \label{eq:contraction-condition}
\end{equation}
for all $\eisv \in \mathcal{X}$, $t \geq 0$. This guarantees that
all trajectories converge exponentially to a unique trajectory at rate
$e^{-\alpha t}$.
\end{definition}

\subsection{Jacobian of the EISV System}

Define $J = \partial f / \partial \eisv$. For the governing equations
\eqref{eq:dE}--\eqref{eq:dV} (using linear $g_I$ mode, see
\Cref{sec:saturation}):

\begin{equation}
J = \begin{pmatrix}
-\alpha - \beta_E S & \alpha & -\beta_E E & 0 \\
0 & -\gamma_I & -k & \beta_I C'(V) \\
0 & 0 & -\mu & -\lambda_2 C'(V) \\
\kappa & -\kappa & 0 & -\delta
\end{pmatrix}
\label{eq:jacobian}
\end{equation}
where $C'(V) = \frac{\partial C}{\partial V} =
\frac{C_{\max} C_1}{2}\operatorname{sech}^2(C_1 V)$ and we have
suppressed the $\deta$-dependent terms (treated as exogenous forcing).

\subsection{Contraction Rate}

\begin{theorem}[EISV Contraction]
\label{thm:contraction}
The EISV system with linear information dynamics ($g_I(I) = \gamma_I I$)
is contracting with rate
\begin{equation}
  \alpha_c = \min\left\{
    \alpha + \beta_E S_{\min},\;
    \gamma_I,\;
    \mu,\;
    \delta
  \right\} - \epsilon(M)
  \label{eq:contraction-rate}
\end{equation}
where $\epsilon(M) \geq 0$ accounts for off-diagonal coupling in the
chosen metric $M \succ 0$. With the diagonal metric
$M = \operatorname{diag}(m_E, m_I, m_S, m_V)$, a sufficient condition
for $\alpha_c > 0$ is:
\begin{equation}
  \min\{\alpha, \gamma_I, \mu, \delta\} >
  \max\!\bigl\{
    \kappa\!\sqrt{\tfrac{m_V}{m_E}},\,
    \beta_I C'_{\max}\!\sqrt{\tfrac{m_I}{m_V}}
  \bigr\}
  \label{eq:contraction-sufficient}
\end{equation}
where $C'_{\max} = C_{\max}C_1/2$ (maximum coherence sensitivity).
\end{theorem}

\begin{proof}
We construct the symmetric part $F = MJ + J^\top M$ for diagonal $M$
and apply Gershgorin's theorem to each row. The diagonal entries
$F_{ii} = 2m_i J_{ii}$ are negative (from $\alpha$, $\gamma_I$,
$\mu$, $\delta$), while off-diagonal entries are bounded by coupling
strengths $\kappa$, $\beta_I C'_{\max}$, and $k$. Choosing metric
weights $m_i$ to make diagonal dominance hold yields the contraction
condition. The full computation with numerical verification for
production parameters is given in \Cref{app:contraction-proof}.
\end{proof}

\begin{remark}[Production Parameters]
\label{rem:production-params}
With production values ($\alpha = 0.42$, $\gamma_I = 0.25$,
$\mu = 0.8$, $\delta = 0.4$, $\kappa = 0.3$, $\beta_I = 0.3$,
$C'_{\max} = 0.5$), the diagonal terms are
$\{0.42, 0.25, 0.8, 0.4\}$ and the coupling bound is dominated by
$\kappa = 0.3$ and $\beta_I C'_{\max} = 0.15$. The full Gershgorin
analysis (\Cref{app:contraction-proof}) with diagonal metric
$M = \operatorname{diag}(0.1, 0.2, 1.0, 0.08)$ yields a certified
contraction rate $\alpha_c = 0.019$. This is conservative: computing the
eigenvalues of $(MJ + J^\top\! M)/(2M)$ directly gives $\alpha_c
\approx 0.13$, and the spectral abscissa of the Jacobian $J$ at
equilibrium is $-0.15$. The Gershgorin bound is tight on Row~2
(the $I$-channel) because the $m_E \alpha$ cross-term competes
with the weak $\gamma_I = 0.25$ diagonal; the actual system
converges roughly $8\times$ faster than the certified bound.
\end{remark}

\subsection{Equilibrium Characterization}

Setting \eqref{eq:dE}--\eqref{eq:dV} to zero and solving:

\begin{proposition}[Equilibrium]
\label{prop:equilibrium}
Under mild conditions ($\alpha > 0$, $\mu > 0$, $\delta > 0$,
$\gamma_I > 0$), the system admits a unique equilibrium
$\eisv^* = (E^*, I^*, S^*, V^*)$ satisfying:
\begin{align}
  V^* &= \frac{\kappa(E^* - I^*)}{\delta} \label{eq:Vstar} \\
  E^* &= \frac{\alpha I^*}{(\alpha + \beta_E S^*)} + O(\norm{\deta}^2)
    \label{eq:Estar} \\
  I^* &= \frac{\beta_I C(V^*)}{\gamma_I + kS^*/\beta_I C(V^*)}
    \label{eq:Istar}
\end{align}
The fixed point is globally exponentially stable by
\Cref{thm:contraction}.
\end{proposition}


% ═══════════════════════════════════════════════════════════
\section{I-Channel Saturation Analysis}
\label{sec:saturation}

\subsection{The Logistic Mode}

The original information dynamics uses a logistic self-regulation term:
\begin{equation}
  g_I^{\text{log}}(I) = \gamma_I I(1-I)
  \label{eq:gi-logistic}
\end{equation}

This creates two potential equilibria for the $I$ subsystem
($dI/dt = 0$):
\begin{equation}
  I^*_{\pm} = \frac{1}{2} \pm \frac{1}{2}\sqrt{1 - \frac{4A}{\gamma_I}}
  \label{eq:I-equilibria}
\end{equation}
where $A = -kS + \beta_I C(V)$ is the net forcing term.

\begin{proposition}[Saturation Criterion]
\label{prop:saturation}
The logistic $I$-dynamics has a boundary saturation pathology when:
\begin{equation}
  A > \frac{\gamma_I}{4}
  \label{eq:saturation-criterion}
\end{equation}
In this regime, $I \to 1$ monotonically (high attractor absorbs all
trajectories), and the system loses its ability to distinguish
integrity levels. The saturation margin is:
\begin{equation}
  m_{\text{sat}} = 1 - \frac{4A}{\gamma_I}
  \label{eq:saturation-margin}
\end{equation}
$m_{\text{sat}} > 0$ indicates safe operation;
$m_{\text{sat}} \leq 0$ indicates saturation.
\end{proposition}

\subsection{The Linear Correction}
\label{sec:linear-correction}

To prevent boundary saturation, we offer an alternative:
\begin{equation}
  g_I^{\text{lin}}(I) = \gamma_I I
  \label{eq:gi-linear}
\end{equation}

This yields a unique equilibrium at:
\begin{equation}
  I^* = \frac{-kS + \beta_I C(V)}{\gamma_I}
  \label{eq:I-star-linear}
\end{equation}
which is always in $(0,1)$ when $\gamma_I >$ max forcing.

\begin{remark}[Mode Selection]
The production system supports both modes via configuration. The
linear mode with $\gamma_I = 0.169$ was tuned to match the logistic
mode's equilibrium at $I^* \approx 0.80$ while eliminating boundary
risk. We recommend the linear mode for production deployments.
\end{remark}

\subsection{Contraction Implications}

Under the linear mode, the Jacobian entry $J_{22} = -\gamma_I$ is
constant, simplifying the contraction analysis. Under the logistic
mode, $J_{22} = -\gamma_I(1 - 2I)$, which changes sign at $I = 0.5$
and approaches zero at the boundaries $I \in \{0, 1\}$---precisely
where contraction is most needed. This provides a principled argument
for the linear correction beyond the saturation pathology alone.


% ═══════════════════════════════════════════════════════════
\section{Adaptive Governor (CIRS v2)}
\label{sec:governor}

Static governance thresholds are fundamentally mismatched to agents
that transition between exploration and integration phases. We
introduce the Contextually-Informed Regulation System version~2
(CIRS~v2), a PID controller that manages governance thresholds as
per-agent state variables.

\subsection{Architecture}

For each agent, CIRS~v2 maintains two adaptive thresholds:
\begin{itemize}
\item $\tau$: coherence threshold (minimum acceptable coherence)
\item $\beta$: risk threshold (maximum acceptable risk)
\end{itemize}
with hard safety bounds $\tau \in [\tau_{\min}, \tau_{\max}]$,
$\beta \in [\beta_{\min}, \beta_{\max}]$ that cannot be overridden
by adaptation.

\subsection{Phase Detection}

The agent's operating phase is detected from EISV trajectory trends
over a sliding window of $w$ steps:

\begin{definition}[Phase Detection]
\label{def:phase}
An agent is in \emph{exploration} phase when at least 2 of 3 signals
are active:
\begin{enumerate}
\item Information growth: $\Delta I / w > \theta_I$ ($\theta_I = 0.008$)
\item Entropy decline: $-\Delta S / w > \theta_S$ ($\theta_S = 0.008$)
\item High complexity: $\bar{\mathcal{C}} > \theta_c$ ($\theta_c = 0.5$)
\end{enumerate}
Otherwise the agent is in \emph{integration} phase.
\end{definition}

\subsection{PID Control Law}

Let $r_\tau^{(p)}$ and $r_\beta^{(p)}$ be the phase-dependent
reference points:
\begin{align}
  \text{Exploration:} \quad r_\tau &= 0.35,\; r_\beta = 0.55 \\
  \text{Integration:} \quad r_\tau &= 0.40,\; r_\beta = 0.60
\end{align}

The error signals are $e_\tau = r_\tau - \tau$ and
$e_\beta = r_\beta - \beta$. The PID update for threshold $\theta
\in \{\tau, \beta\}$ is:
\begin{equation}
  u_\theta = K_p e_\theta + K_i \int_0^t e_\theta\,dt'
  + K_d \cdot d_p \cdot \frac{de_\theta}{dt}
  \label{eq:pid}
\end{equation}
with gains $K_p = 0.05$, $K_i = 0.005$, $K_d = 0.10$ and
phase-dependent derivative factor:
\begin{equation}
  d_p = \begin{cases}
    0.5 & \text{exploration (tolerate oscillation)} \\
    1.0 & \text{integration (full damping)}
  \end{cases}
  \label{eq:d-factor}
\end{equation}

\begin{remark}[The D-term IS the damping]
The derivative gain $K_d = 0.10$ is intentionally the strongest term.
Oscillation in governance decisions produces large $de/dt$, which
produces large corrective updates. The system self-stabilizes without
requiring a separate oscillation detector.
\end{remark}

\subsection{Integral Wind-Up Protection}

The integral term is clamped:
\begin{equation}
  \left|\int_0^t e_\theta\,dt'\right| \leq I_{\max} = 0.10
  \label{eq:windup}
\end{equation}
with zero-crossing reset: when $\operatorname{sign}(e_\theta)$ changes,
the integral accumulator is zeroed.

\subsection{Threshold Decay}

When the oscillation index $\text{OI} < 0.5$ (stable), thresholds
decay toward defaults at rate $\lambda_d = 0.01$ per update:
\begin{equation}
  \theta \leftarrow \theta + \lambda_d(\theta_{\text{default}} - \theta)
  \label{eq:decay}
\end{equation}

\subsection{Multi-Agent Coordination}

Neighbor pressure tightens thresholds when nearby agents are
experiencing difficulty:
\begin{equation}
  \Delta\tau_{\text{nb}} = \frac{\alpha_n}{|\mathcal{N}|}
  \sum_{j \in \mathcal{N}} \mathbb{1}[C_j < \tau_j]
  \label{eq:neighbor}
\end{equation}
where $\mathcal{N}$ is the set of agents with similar coherence
profiles and $\alpha_n$ is the neighbor coupling strength.

\subsection{Governance Decision}

The adaptive governor produces a binary verdict:
\begin{equation}
  \text{verdict} = \begin{cases}
    \texttt{proceed} & C(V) \!\geq\! \tau \;\wedge\; R \!<\! \beta \\
    \texttt{pause} & \text{otherwise}
  \end{cases}
  \label{eq:verdict}
\end{equation}
where $R$ is the risk score derived from $\Phi$ (\Cref{eq:phi}).

\begin{remark}[Two-Tier vs.\ Three-Tier Governance]
An earlier design used three tiers: approve/revise/reject.
In production, the middle \texttt{revise} state was ambiguous---agents
receiving a ``revise'' verdict had no actionable signal for \emph{what}
to revise, leading to oscillation between revise and approve as
marginal coherence values crossed the threshold. The two-tier design
eliminates this source of instability: agents either proceed (the
system is contracting normally) or pause (wait for human or
environmental input to shift the state). The PID governor's continuous
threshold adaptation provides the nuance that the discrete middle tier
attempted to capture.
\end{remark}


% ═══════════════════════════════════════════════════════════
\section{Ethical Drift Vector}
\label{sec:ethical-drift}

\subsection{Motivation}

``Ethical drift'' in prior work has been an abstract concept---the
idea that an agent might gradually deviate from its intended values.
We ground this in four measurable signals, each bounded in $[0,1]$:

\begin{definition}[Ethical Drift Vector]
\label{def:drift}
\begin{equation}
  \deta = \begin{pmatrix}
    \eta_{\text{cal}} \\
    \eta_{\text{cpx}} \\
    \eta_{\text{coh}} \\
    \eta_{\text{stab}}
  \end{pmatrix} \in [0,1]^4
  \label{eq:drift-vector}
\end{equation}
where:
\begin{itemize}
\item $\eta_{\text{cal}}$: \textbf{Calibration}---gap
  between predicted and actual correctness,
  $|\hat{p} - p_{\text{act}}|$.
\item $\eta_{\text{cpx}}$: \textbf{Complexity}---discrepancy
  between derived and self-reported task complexity,
  $|C_{\text{der}} - C_{\text{self}}|$.
\item $\eta_{\text{coh}}$: \textbf{Coherence}---departure
  from EMA baseline, $|C(V)_t - \bar{C}(V)|$.
\item $\eta_{\text{stab}}$: \textbf{Stability}---$1 -
  \text{consistency}$ over a 20-step decision window.
\end{itemize}
\end{definition}

\subsection{Baseline Tracking}

Each component is tracked against an exponential moving average (EMA)
baseline with smoothing parameter $\alpha_{\text{ema}} = 0.1$:
\begin{equation}
  \bar{x}_{t+1} = (1 - \alpha_{\text{ema}})\bar{x}_t
  + \alpha_{\text{ema}} x_t
  \label{eq:ema}
\end{equation}

A warmup dampening factor prevents spurious drift detection from
uninitialized baselines:
\begin{equation}
  w(n) = \min\left(\frac{n}{n_{\text{warmup}}}, 1\right),
  \quad n_{\text{warmup}} = 5
  \label{eq:warmup}
\end{equation}
applied as $\deta_{\text{effective}} = w(n) \cdot \deta$.

\subsection{Coupling to Dynamics}

The drift norm $\norm{\deta}$ enters the dynamics at two points:
\begin{enumerate}
\item \textbf{Energy} \eqref{eq:dE}: $+\gamma_E\norm{\deta}^2$
  drives energy up (agents work harder when drifting---a destabilizing
  positive feedback).
\item \textbf{Entropy} \eqref{eq:dS}:
  $+\lambda_1(\Theta)\norm{\deta}^2$ increases uncertainty (drifting
  agents become less certain---a stabilizing negative feedback through
  the governance verdict).
\end{enumerate}

The net effect is that ethical drift creates a governance response:
increased uncertainty triggers conservative decisions, while the energy
spike draws attention to the drifting agent.

\begin{remark}[Energy Spike Containment]
The $+\gamma_E\norm{\deta}^2$ term in \eqref{eq:dE} creates a
transient energy surplus in drifting agents. Left unchecked, this
could produce high-energy, low-integrity states (``manic'' agents
acting confidently while miscalibrated). Two mechanisms contain this:
(i)~the entropy coupling $+\lambda_1\norm{\deta}^2$ in \eqref{eq:dS}
simultaneously raises $S$, which erodes $E$ through $-\beta_E ES$ in
\eqref{eq:dE} and drives the objective $\Phi$ below the governance
threshold, triggering a pause verdict; (ii)~the adaptive governor
(\Cref{sec:governor}) tightens $\tau$ when coherence drops, which
happens as $V$ responds to the $E$-$I$ divergence. With production
parameters, $\gamma_E = 0.05$ while $\lambda_1 \in [0.05, 0.20]$,
so the entropy response dominates the energy excitation for any
$\norm{\deta} > 0$.
\end{remark}


% ═══════════════════════════════════════════════════════════
\section{Stochastic Extensions}
\label{sec:stochastic}

In production, the dynamics are subject to stochastic perturbations
from unpredictable task inputs. We model this as an It\^{o} SDE:
\begin{equation}
  d\eisv = f(\eisv, t)\,dt + G(\eisv, t)\,dW_t
  \label{eq:sde}
\end{equation}
where $W_t$ is a standard Wiener process and
$G(\eisv, t) \in \R^{4 \times 4}$ is the diffusion matrix.

\begin{theorem}[Mean-Square Contraction]
\label{thm:stochastic}
If the deterministic system is contracting with rate $\alpha_c$ under
metric $M$, and the diffusion satisfies
\begin{equation}
  \operatorname{tr}(G^\top M G) \leq \sigma_{\max}^2
  \label{eq:diffusion-bound}
\end{equation}
then the stochastic system satisfies:
\begin{multline}
  \mathbb{E}\bigl[\norm{\eisv_1(t) - \eisv_2(t)}_M^2\bigr] \leq \\
  e^{-2\alpha_c t}\norm{\eisv_1(0) - \eisv_2(0)}_M^2
  + \frac{\sigma_{\max}^2}{2\alpha_c}
  \label{eq:mean-square}
\end{multline}
The steady-state variance is bounded by
$\sigma_{\max}^2 / (2\alpha_c)$.
\end{theorem}

\begin{remark}[Verification for Production Noise Model]
In production, the diffusion matrix is
$G = \operatorname{diag}(0, 0, \sigma, 0)$ with $\sigma = 0.02$
(additive noise on the entropy channel only). For any diagonal metric
$M = \operatorname{diag}(m_E, m_I, m_S, m_V)$:
\[
  \operatorname{tr}(G^\top M G) = m_S \sigma^2
\]
Setting $m_S = 1$ (natural scaling), $\sigma_{\max}^2 = (0.02)^2 =
4 \times 10^{-4}$. With the certified contraction rate
$\alpha_c = 0.019$ (\Cref{app:contraction-proof}), the steady-state
variance bound from \eqref{eq:mean-square} is:
\[
  \frac{\sigma_{\max}^2}{2\alpha_c} = \frac{4 \times 10^{-4}}{0.038}
  \approx 0.011
\]
Using the tighter eigenvalue rate $\alpha_c \approx 0.13$
(Remark~\ref{rem:production-params}) would yield $1.5 \times 10^{-3}$,
but the conservative Gershgorin bound already shows the noise is
negligible. This is one order of magnitude smaller than typical EISV signal
excursions ($\sim 0.1$--$0.3$), confirming that stochastic noise does
not compromise governance decisions. The result follows directly from
Pham et~al.~\cite{pham2009stochastic}.
\end{remark}


% ═══════════════════════════════════════════════════════════
\section{Multi-Agent Network}
\label{sec:network}

When multiple agents operate simultaneously, their governance states
are coupled through shared information and the neighbor pressure
mechanism of CIRS~v2.

\subsection{Network Topology}

Consider $N$ agents with states $\eisv_i$, $i = 1, \ldots, N$,
connected by a graph $\mathcal{G} = (\mathcal{V}, \mathcal{E})$ with
Laplacian $L$. The coupled dynamics are:
\begin{equation}
  \dot{\eisv}_i = f(\eisv_i, t)
  + \epsilon_c \sum_{j \in \mathcal{N}_i} H(\eisv_j - \eisv_i)
  \label{eq:network}
\end{equation}
where $H \in \R^{4 \times 4}$ is a coupling matrix and
$\epsilon_c > 0$ is the coupling strength.

\begin{theorem}[Network Synchronization]
\label{thm:network}
If each agent's isolated dynamics is contracting with rate $\alpha_c$,
the network synchronizes (all agents converge to a common trajectory)
provided:
\begin{equation}
  \alpha_c > \lambda_{\max}(L) \cdot \lambda_{\max}(H)
  \label{eq:sync-condition}
\end{equation}
where $\lambda_{\max}$ denotes the largest eigenvalue.
\end{theorem}


% ═══════════════════════════════════════════════════════════
\section{Experiments}
\label{sec:experiments}

\subsection{Production Deployment}

The UNITARES framework is deployed as a Model Context Protocol (MCP)
server that provides governance services to autonomous AI agents. Each
agent connects via standard MCP transport and receives governance
verdicts on every action. The system has been in continuous production
since December 2025.

\begin{table}[htbp]
\centering
\small
\caption{Production Deployment Summary}
\label{tab:deployment}
\begin{tabular}{@{}lr@{}}
\toprule
\textbf{Metric} & \textbf{Value} \\
\midrule
Registered agents & 903 \\
Active ($>5$ updates) & 75 \\
Max updates (single agent) & 48{,}668 \\
Audit events & 198{,}333 \\
Knowledge discoveries & 500 \\
Dialectic sessions & 66 \\
Deployment duration & 69 days \\
\bottomrule
\end{tabular}
\end{table}

The 903 registered agents include both long-running production agents
(5 agents with $>1{,}000$ updates, peak at 48{,}668) and shorter-lived
agents from development and testing cycles. The top two agents have
accumulated 48{,}668 and 44{,}529 updates respectively, providing
substantial trajectory data for convergence analysis.

\subsection{EISV Equilibrium Validation}

For the 75 agents with sufficient update history ($>5$ updates), the
observed equilibrium statistics are:

\begin{table}[htbp]
\centering
\small
\caption{Observed EISV Equilibrium (75 Active Agents)}
\label{tab:eisv-observed}
\begin{tabular}{@{}lrrr@{}}
\toprule
\textbf{Variable} & \textbf{Mean} & \textbf{Std} & \textbf{Range} \\
\midrule
$E$ & 0.766 & 0.087 & $[0.60, 1.00]$ \\
$I$ & 0.877 & 0.111 & $[0.59, 1.00]$ \\
$S$ & 0.082 & 0.045 & $[0.01, 0.20]$ \\
$V$ & $-0.033$ & 0.028 & $[-0.09, 0.01]$ \\
$C(V)$ & 0.483 & 0.012 & $[0.46, 0.50]$ \\
\bottomrule
\end{tabular}
\end{table}

\Cref{tab:eisv-observed} summarizes the observed equilibrium.
These values validate several theoretical predictions:
\begin{enumerate}
\item \textbf{$I > E$:} Mean integrity (0.877) exceeds mean energy
  (0.766), confirming the system's conservative bias
  (\Cref{fig:ei-scatter}). The negative mean $V = -0.033$ (integrity
  surplus) is consistent with this.
\item \textbf{Low entropy:} Mean $S = 0.082$ with strong decay ($\mu
  = 0.8$) indicates effective uncertainty management.
\item \textbf{Coherence at 0.49:} The observed mean $C = 0.483$
  (\Cref{fig:coherence-hist}) confirms
  Remark~\ref{rem:operating-range}---the system honestly reports
  near-midpoint coherence rather than inflating the signal.
\item \textbf{$V$ confined to $[-0.1, 0.1]$:} 100\% of active agents
  have $V$ within $[-0.089, 0.007]$ (\Cref{fig:sv-scatter}), validating
  the damping analysis.
\end{enumerate}

\begin{figure}[t]
\centering
\includegraphics[width=\columnwidth]{fig1_ei_scatter}
\caption{Energy--Integrity scatter plot for 75 active agents, colored by
dynamical regime. Nearly all agents lie above the $I = E$ diagonal,
confirming the system's conservative bias ($\bar{V} < 0$). Point size
reflects log update count.}
\label{fig:ei-scatter}
\end{figure}

\begin{figure}[t]
\centering
\includegraphics[width=\columnwidth]{fig2_sv_scatter}
\caption{Entropy--Void operating region. All agents have $V \in [-0.089,
0.007]$, well within the $[-0.1, 0.1]$ predicted by damping analysis.
Entropy remains low ($S < 0.20$) due to strong decay $\mu = 0.8$.}
\label{fig:sv-scatter}
\end{figure}

\begin{figure}[t]
\centering
\includegraphics[width=\columnwidth]{fig3_coherence_hist}
\caption{Distribution of coherence $C(V)$ across active agents. The mean
$\bar{C} = 0.483$ lies near the theoretical midpoint ($C = 0.5$),
consistent with small $|V|$ and the honest-reporting principle.}
\label{fig:coherence-hist}
\end{figure}

\subsection{I-Channel Saturation}

Of 903 agents, 19 have $I \geq 0.999$ (boundary saturation). All 19
are in the CONVERGENCE regime with relatively few updates, indicating
that saturation occurs in agents that receive strong positive forcing
(high coherence, low entropy) without sufficient self-regulation.

For the 75 active agents, the saturation margin under logistic mode
is:
\begin{align*}
  m_{\text{sat}} &= 1 - \frac{4(-kS^* + \beta_I C^*)}{\gamma_I} \\
  &= 1 - \frac{4(-0.10 \cdot 0.082 + 0.30 \cdot 0.483)}{0.25} \\
  &= 1 - \tfrac{0.569}{0.25} = -1.28
\end{align*}
This \emph{negative} margin confirms the saturation criterion
(\Cref{prop:saturation}): under the logistic mode with production
parameters, the forcing term exceeds $\gamma_I/4$, and only the
clamping at $I = 1$ prevents unbounded growth
(\Cref{fig:saturation-margin}). The linear mode eliminates this
issue by construction.

\begin{figure}[t]
\centering
\includegraphics[width=\columnwidth]{fig4_saturation_margin}
\caption{Saturation margin $m_{\mathrm{sat}} = 1 - 4A/\gamma_I$ by
regime. All agents fall below zero (horizontal line), confirming that
under logistic damping, every agent's forcing exceeds the maximum
logistic restoring force. The linear correction
(\Cref{sec:linear-correction}) removes this boundary pathology.}
\label{fig:saturation-margin}
\end{figure}

\subsection{Regime Distribution}

Agents distribute across three dynamical regimes based on EISV
trajectory trends:

\begin{table}[htbp]
\centering
\small
\caption{Regime Distribution and EISV Profiles (Active Agents)}
\label{tab:regimes}
\begin{tabular}{@{}lrrrrr@{}}
\toprule
\textbf{Regime} & $n$ & $\bar{E}$ & $\bar{I}$ & $\bar{S}$ & $\bar{C}$ \\
\midrule
Convergence & 24 & .838 & .961 & .052 & .474 \\
Exploration & 22 & .766 & .926 & .090 & .477 \\
Divergence & 29 & .706 & .770 & .100 & .496 \\
\bottomrule
\end{tabular}
\end{table}

The regime profiles (\Cref{fig:regime-profiles}) show clear
separation: convergent agents have the highest integrity and lowest
entropy; divergent agents have the lowest integrity and highest
entropy. Coherence is slightly
higher in the divergent regime ($\bar{C} = 0.496$) because these
agents have smaller $|V|$, placing them nearer the coherence
function's midpoint.

\begin{figure}[t]
\centering
\includegraphics[width=\columnwidth]{fig5_regime_profiles}
\caption{Mean EISV component values by dynamical regime. Convergent
agents show the highest $I$ and lowest $S$; divergent agents show the
reverse pattern. The $|V|$ is small across all regimes, confirming
effective damping.}
\label{fig:regime-profiles}
\end{figure}

\subsection{Knowledge Graph and Dialectic}

The governance system has accumulated 500 knowledge graph discoveries
classified by severity: 238 low, 154 medium, 49 high, and 4 critical.
By type, the largest categories are insights (183), notes (125),
improvements (94), and bugs found (48).

The dialectic protocol---a structured thesis/antithesis/synthesis
process for resolving governance disputes---has conducted 66 sessions:
39 recovery sessions, 15 reviews, and 12 explorations. Of these,
21 reached resolution, while 45 terminated without agreement,
suggesting that the dialectic process is more effective for recovery
(where the system has clear signals) than for open-ended review.

\begin{figure}[t]
\centering
\includegraphics[width=\columnwidth]{fig6_discovery_types}
\caption{Knowledge graph discovery classification by type. Insights
and notes dominate, with bugs and improvements constituting the
actionable governance outputs.}
\label{fig:discovery-types}
\end{figure}

\subsection{Limitations of Empirical Validation}

The production data has three limitations that bound the strength of
our empirical claims:
\begin{enumerate}
\item \textbf{Parameter evolution:} Dynamics parameters, governor
  thresholds, and the I-channel mode were refined iteratively during
  deployment. The reported EISV statistics are a snapshot across agents
  that experienced different parameter regimes. A controlled A/B
  comparison between logistic and linear modes has not been conducted.
\item \textbf{Governor recency:} The adaptive governor (CIRS~v2) was
  deployed recently; the audit log does not yet contain sufficient
  pre/post data for a rigorous before/after comparison.
\item \textbf{Population heterogeneity:} The 903 agents include
  long-running production agents, short-lived test agents, and agents
  from early development with since-changed parameters. We report
  results for the 75 agents with $>5$ updates to mitigate this, but
  acknowledge that even this filtered set spans multiple configuration
  epochs.
\end{enumerate}


% ═══════════════════════════════════════════════════════════
\section{Discussion}
\label{sec:discussion}

\subsection{Design Principle: Adjust Expectations, Not Measurements}
\label{sec:honest}

A core design principle of UNITARES is that observations are never
transformed to appear healthier. When production coherence averages
$C \approx 0.49$ rather than the ``expected'' $C \approx 0.7$, the
system reports 0.49. The phase-aware governor (\Cref{sec:governor})
addresses the gap by adjusting \emph{thresholds} (expectations), not
\emph{signals} (measurements).

This principle---\textbf{adjust expectations, not measurements}---has
three practical consequences:
\begin{enumerate}
\item \textbf{Calibration integrity:} Drift detection
  (\Cref{sec:ethical-drift}) compares raw signals against baselines.
  Inflating measurements would mask the very drift the system is
  designed to detect.
\item \textbf{Operator trust:} Dashboard values correspond to actual
  system state. An operator seeing $C = 0.49$ knows the agent is
  operating conservatively, not that a scaling factor was applied.
\item \textbf{Contraction validity:} The stability proofs in
  \Cref{sec:contraction} depend on the true state. Transforming
  measurements would invalidate the contraction analysis by
  introducing an unmodeled observation function.
\end{enumerate}

\subsection{Parameter Sensitivity}

The system's behavior is most sensitive to four parameters:

\paragraph{Convergence rate $\alpha$.}
Controls how quickly $E$ tracks $I$. Production uses $\alpha = 0.42$;
earlier theoretical work used $\alpha = 0.5$. Larger $\alpha$
increases diagonal dominance in the Jacobian (improving contraction)
but makes the energy channel less responsive to ethical drift.

\paragraph{Coherence coupling $\beta_I$.}
Determines how strongly coherence restores integrity. The theoretical
optimum for contraction ($\beta_I = 0.05$, minimizing off-diagonal
coupling) conflicts with the operational need for coherence feedback
($\beta_I = 0.30$, enabling meaningful recovery from entropy-driven
degradation). Production uses the larger value; contraction is
maintained because $\beta_I C'_{\max} = 0.15$ remains below the
diagonal terms.

\paragraph{Self-regulation $\gamma_I$.}
Determines the I-channel's intrinsic damping. In logistic mode,
$\gamma_I = 0.25$ yields $m_{\text{sat}} = -1.23$ at production
operating points (\Cref{app:saturation})---all agents are saturated,
relying on clamping rather than dynamics to bound $I$. In linear mode,
$\gamma_I = 0.169$ was tuned to give $I^* \approx 0.80$, well within
$(0,1)$, eliminating boundary dependence entirely.

\paragraph{Void damping $\delta$.}
At $\delta = 0.4$, the void channel has a time constant of
$\tau_V = 1/\delta = 2.5$ update cycles. This confines $V$ to
$[-0.1, 0.1]$ in production (\Cref{rem:operating-range}). Reducing
$\delta$ would widen the coherence operating range but slow the
system's response to $E$-$I$ imbalances.

\subsection{Boundary Clamping and Non-Smoothness}

The EISV state is clamped to $\mathcal{X}$ at each integration step,
introducing non-smoothness at domain boundaries. We address this in
three ways:

\paragraph{Interior operation.}
The epistemic humility floor ($S_{\min} = 0.001$) and the contraction
dynamics ensure that trajectories remain in the interior of
$\mathcal{X}$ during normal operation. In production, only 19 of 903
agents (2.1\%) contact the $I = 1$ boundary, all in the CONVERGENCE
regime with few updates, confirming that interior operation is the norm.

\paragraph{Projected dynamical systems.}
When boundary contact does occur, the clamping operation implements a
projection onto the constraint set $\mathcal{X}$. Projected dynamical
systems preserve contraction properties when the constraint set is
convex~\cite{lohmiller1998contraction}, which $\mathcal{X}$ (a
hyperrectangle) is.

\paragraph{Practical monitoring.}
The saturation diagnostic (\Cref{sec:saturation}) monitors the
I-channel's distance from the boundary. When $m_{\text{sat}} < 0.1$,
the system logs a warning and the adaptive governor tightens
thresholds pre-emptively, preventing boundary contact rather than
relying on clamping to maintain invariance.

\subsection{Limitations}

\begin{enumerate}
\item The coherence function $\cv$ creates a nonlinear feedback
  loop whose gain depends on the operating point (maximum at
  $V = 0$, vanishing at $|V| \gg 0$). The contraction analysis
  uses $C'_{\max}$ as a worst-case bound, which is conservative.
\item The ethical drift vector's four components are currently
  equally weighted; adaptive weighting based on domain context
  is future work.
\item The multi-agent network analysis assumes fixed topology;
  production agent populations are dynamic with agents joining
  and leaving.
\item The PID governor gains ($K_p$, $K_i$, $K_d$) were tuned
  empirically; optimal gain selection via, e.g., Ziegler-Nichols
  or LQR methods is future work.
\end{enumerate}

\subsection{Connections to Thermodynamics}

The EISV framework is thermodynamically \emph{inspired} rather than
thermodynamically \emph{derived}. The analogy is:
\begin{center}
\small
\begin{tabular}{@{}ll@{}}
\toprule
\textbf{Thermodynamics} & \textbf{UNITARES} \\
\midrule
Internal energy $U$ & Energy $E$ \\
Entropy $S$ & Entropy $S$ \\
Helmholtz free energy $F$ & Void $V$ (approximately) \\
Temperature $T$ & \emph{No direct analog} \\
\bottomrule
\end{tabular}
\end{center}
We do not claim thermodynamic \emph{derivation} from first
principles; the equations are phenomenological models whose
stability properties are established mathematically rather than
thermodynamically.


% ═══════════════════════════════════════════════════════════
\section{Conclusion}
\label{sec:conclusion}

UNITARES demonstrates that autonomous agent governance can be
formalized as a continuous dynamical system with provable stability
properties. The six contributions---EISV dynamics, contraction
analysis, saturation resolution, adaptive governor, concrete
ethical drift, and production validation---together provide a
framework that is both mathematically rigorous and practically
deployable.

The key insight is that governance thresholds should be living state,
not dead configuration. By treating thresholds as adaptive variables
governed by PID control with phase-aware reference tracking, the
system naturally accommodates the tension between exploration and
safety without sacrificing formal guarantees.

Production deployment with 903 agents over 69 days validates the
theoretical predictions: observed EISV equilibria match analytical
bounds, the void variable is confined to the predicted $[-0.1, 0.1]$
range in 100\% of active agents, and the saturation analysis correctly
identifies the 19 agents that reach the $I = 1$ boundary.

As autonomous AI agents become more prevalent and capable, the need
for governance mechanisms with formal properties---not just
heuristics---becomes acute. UNITARES provides one path forward:
treat the agent's relationship to its own reliability as a dynamical
system, prove that system is stable, and let the mathematics do the
governing.


% ═══════════════════════════════════════════════════════════
% REFERENCES
% ═══════════════════════════════════════════════════════════
\bibliographystyle{plainnat}

\begin{thebibliography}{99}

\bibitem[Lohmiller and Slotine(1998)]{lohmiller1998contraction}
W.~Lohmiller and J.-J.~E. Slotine.
\newblock On contraction analysis for non-linear systems.
\newblock \emph{Automatica}, 34(6):683--696, 1998.

\bibitem[Slotine(2003)]{slotine2003modular}
J.-J.~E. Slotine.
\newblock Modular stability tools for distributed computation and control.
\newblock \emph{International Journal of Adaptive Control and Signal Processing},
  17(6):397--416, 2003.

\bibitem[Guo et~al.(2017)]{guo2017calibration}
C.~Guo, G.~Pleiss, Y.~Sun, and K.~Q. Weinberger.
\newblock On calibration of modern neural networks.
\newblock In \emph{ICML}, pages 1321--1330, 2017.

\bibitem[Niculescu-Mizil and Caruana(2005)]{niculescu2005predicting}
A.~Niculescu-Mizil and R.~Caruana.
\newblock Predicting good probabilities with supervised learning.
\newblock In \emph{ICML}, pages 625--632, 2005.

\bibitem[\AA{}str\"{o}m and Murray(2008)]{astrom2006advanced}
K.~J. \AA{}str\"{o}m and R.~M. Murray.
\newblock \emph{Feedback Systems: An Introduction for Scientists and Engineers}.
\newblock Princeton University Press, 2008.

\bibitem[Bai et~al.(2022)]{bai2022constitutional}
Y.~Bai, S.~Kadavath, S.~Kundu, et~al.
\newblock Constitutional {AI}: Harmlessness from {AI} feedback.
\newblock \emph{arXiv preprint arXiv:2212.08073}, 2022.

\bibitem[Ouyang et~al.(2022)]{ouyang2022training}
L.~Ouyang, J.~Wu, X.~Jiang, et~al.
\newblock Training language models to follow instructions with human feedback.
\newblock In \emph{NeurIPS}, 2022.

\bibitem[Rebedea et~al.(2023)]{rebedea2023nemo}
T.~Rebedea, R.~Dinu, M.~Sreedhar, C.~Parisien, and J.~Cohen.
\newblock {NeMo} {G}uardrails: A toolkit for controllable and safe {LLM}
  applications with programmable rails.
\newblock In \emph{EMNLP (Demo)}, 2023.

\bibitem[Ravindran(2025)]{ravindran2025moral}
S.~Ravindran.
\newblock A predictive framework for {AI} value alignment and drift prevention.
\newblock \emph{arXiv preprint arXiv:2510.04073}, 2025.

\bibitem[Ames et~al.(2019)]{ames2019control}
A.~D. Ames, S.~Coogan, M.~Egerstedt, G.~Notomista, K.~Sreenath, and
  P.~Tabuada.
\newblock Control barrier functions: Theory and applications.
\newblock In \emph{European Control Conference (ECC)}, pages 3420--3431, 2019.

\bibitem[Pham et~al.(2009)]{pham2009stochastic}
Q.-C. Pham, N.~Tabareau, and J.-J.~E. Slotine.
\newblock A contraction theory approach to stochastic incremental stability.
\newblock \emph{IEEE Transactions on Automatic Control}, 54(4):816--820, 2009.

\bibitem[It\^{o}(1944)]{ito1944}
K.~It\^{o}.
\newblock Stochastic integral.
\newblock \emph{Proceedings of the Imperial Academy}, 20(8):519--524, 1944.

\bibitem[Landauer(1961)]{landauer1961}
R.~Landauer.
\newblock Irreversibility and heat generation in the computing process.
\newblock \emph{IBM Journal of Research and Development}, 5(3):183--191, 1961.

\end{thebibliography}


% ═══════════════════════════════════════════════════════════
% APPENDIX
% ═══════════════════════════════════════════════════════════
\appendix

\section{Parameter Tables}
\label{app:parameters}

\begin{table}[htbp]
\centering
\footnotesize
\caption{EISV Dynamics Parameters (Production Values)}
\label{tab:params}
\begin{tabular}{@{}llrl@{}}
\toprule
\textbf{Parameter} & \textbf{Symbol} & \textbf{Val.} & \textbf{Role} \\
\midrule
\multicolumn{4}{@{}l}{\textit{Energy channel}} \\
Convergence & $\alpha$ & 0.42 & $E$ tracks $I$ \\
Entropy erosion & $\beta_E$ & 0.10 & $S$ erodes $E$ \\
Drift excitation & $\gamma_E$ & 0.05 & Drift raises $E$ \\
\midrule
\multicolumn{4}{@{}l}{\textit{Information channel}} \\
Entropy degrad. & $k$ & 0.10 & $S$ degrades $I$ \\
Coherence boost & $\beta_I$ & 0.30 & $C(V)$ restores $I$ \\
Self-regulation & $\gamma_I$ & 0.25 & Logistic damping \\
Linear mode & $\gamma_I^{\text{lin}}$ & 0.169 & Linear damping \\
\midrule
\multicolumn{4}{@{}l}{\textit{Entropy channel}} \\
Natural decay & $\mu$ & 0.80 & Dissipation \\
Drift injection & $\lambda_{1}$ & 0.30 & Base (adaptive) \\
Coherence red. & $\lambda_{2}$ & 0.05 & $C(V)$ reduces $S$ \\
Complexity & $\beta_c$ & 0.15 & Task $\to$ entropy \\
Noise & $\sigma$ & 0.02 & Stochastic term \\
\midrule
\multicolumn{4}{@{}l}{\textit{Void channel}} \\
Coupling & $\kappa$ & 0.30 & $E$-$I$ gap \\
Damping & $\delta$ & 0.40 & Dissipation \\
\midrule
\multicolumn{4}{@{}l}{\textit{Coherence}} \\
Maximum & $C_{\max}$ & 1.0 & Upper bound \\
Sensitivity & $C_1$ & 1.0 & Tanh slope \\
\bottomrule
\end{tabular}
\end{table}

\begin{table}[htbp]
\centering
\footnotesize
\caption{CIRS v2 Adaptive Governor Parameters}
\label{tab:governor}
\begin{tabular}{@{}llrl@{}}
\toprule
\textbf{Parameter} & \textbf{Symbol} & \textbf{Val.} & \textbf{Role} \\
\midrule
Proportional & $K_p$ & 0.05 & Gentle correction \\
Integral & $K_i$ & 0.005 & Drift correction \\
Derivative & $K_d$ & 0.10 & Damping \\
Integral limit & $I_{\max}$ & 0.10 & Wind-up cap \\
\midrule
\multicolumn{4}{@{}l}{\textit{Reference points}} \\
Expl.\ $\tau$ ref & $r_\tau^{(e)}$ & 0.35 & Forgiving \\
Expl.\ $\beta$ ref & $r_\beta^{(e)}$ & 0.55 & Forgiving \\
Intg.\ $\tau$ ref & $r_\tau^{(i)}$ & 0.40 & Strict \\
Intg.\ $\beta$ ref & $r_\beta^{(i)}$ & 0.60 & Strict \\
\midrule
\multicolumn{4}{@{}l}{\textit{Safety bounds}} \\
$\tau$ floor/ceil & $\tau$ & 0.25/0.75 & Hard bounds \\
$\beta$ floor/ceil & $\beta$ & 0.20/0.70 & Hard bounds \\
\bottomrule
\end{tabular}
\end{table}


\begin{table}[htbp]
\centering
\footnotesize
\caption{Ethical Drift Vector Parameters}
\label{tab:drift}
\begin{tabular}{@{}llrl@{}}
\toprule
\textbf{Parameter} & \textbf{Symbol} & \textbf{Val.} & \textbf{Role} \\
\midrule
\multicolumn{4}{@{}l}{\textit{Baseline tracking}} \\
EMA smoothing & $\alpha_{\text{ema}}$ & 0.1 & Update rate \\
Warmup steps & $n_w$ & 5 & Drift suppression \\
\midrule
\multicolumn{4}{@{}l}{\textit{Stability measurement}} \\
Decision window & $w_s$ & 20 & Consistency \\
Transition pen. & --- & /flip & Instability \\
\midrule
\multicolumn{4}{@{}l}{\textit{Coupling to dynamics}} \\
Energy excit. & $\gamma_E$ & 0.05 & Drift $\to dE$ \\
Entropy inj. & $\lambda_1$ & .05--.20 & Drift $\to dS$ \\
\bottomrule
\end{tabular}
\end{table}


\section{Proof of Contraction (Full)}
\label{app:contraction-proof}

We prove \Cref{thm:contraction} by constructing a diagonal metric
$M = \operatorname{diag}(m_E, m_I, m_S, m_V) \succ 0$ and showing
that the generalized Jacobian condition \eqref{eq:contraction-condition}
holds throughout the state space $\mathcal{X}$.

\subsection*{Step 1: Symmetric Part of $MJ + J^\top M$}

For diagonal $M$, the matrix $F = MJ + J^\top M$ has entries:
\begin{align*}
  F_{ii} &= 2 m_i J_{ii} \\
  F_{ij} &= m_i J_{ij} + m_j J_{ji}, \quad i \neq j
\end{align*}

The diagonal entries of $F$ (using linear $g_I$ mode) are:
\begin{align}
  F_{11} &= -2m_E(\alpha + \beta_E S) \leq -2m_E \alpha
    \label{eq:F11} \\
  F_{22} &= -2m_I \gamma_I
    \label{eq:F22} \\
  F_{33} &= -2m_S \mu
    \label{eq:F33} \\
  F_{44} &= -2m_V \delta
    \label{eq:F44}
\end{align}
where \eqref{eq:F11} uses $S \geq S_{\min} \geq 0$.

The nonzero off-diagonal entries are:
\begin{align}
  F_{12} = F_{21} &= m_E \alpha \label{eq:F12} \\
  F_{13} &= -m_E \beta_E E \leq 0 \label{eq:F13} \\
  F_{23} &= -m_I k \label{eq:F23} \\
  F_{24} &= m_I \beta_I C'(V) \label{eq:F24} \\
  F_{34} &= -m_V \lambda_2 C'(V) \label{eq:F34} \\
  F_{14} = F_{41} &= m_V \kappa \label{eq:F14}
\end{align}
where $F_{13}$ and $F_{34}$ are negative (aiding contraction)
and are bounded conservatively by their absolute values in the
Gershgorin analysis below.

Note: For diagonal $M$, $F_{12} = F_{21} = m_E J_{12} + m_I J_{21}
= m_E \alpha$.

\subsection*{Step 2: Gershgorin Circle Theorem}

For $F + 2\alpha_c M \preceq 0$, it suffices by Gershgorin that
for each row~$i$:
\begin{equation}
  F_{ii} + 2\alpha_c m_i + \!\sum_{j \neq i}\! |F_{ij}| \leq 0
  \label{eq:gershgorin}
\end{equation}

Bounding $|F_{13}| \leq m_E\beta_E$ conservatively (its sign aids
contraction), the four row conditions are:

\paragraph{Row 1 ($E$):}
\begin{equation}
  m_E(-\alpha + \alpha_c + \beta_E) + m_V\kappa \leq 0
  \label{eq:gersh-row1}
\end{equation}

\paragraph{Row 2 ($I$):}
\begin{equation}
  \begin{split}
  m_I(-2\gamma_I + 2\alpha_c + k &+ \beta_I C'_{\max}) \\
  &+ m_E\alpha \leq 0
  \end{split}
  \label{eq:gersh-row2}
\end{equation}

\paragraph{Row 3 ($S$):}
\begin{equation}
  \begin{split}
  -2m_S\mu + 2\alpha_c m_S &+ m_E\beta_E \\
  &+ m_I k + m_V\lambda_2 C'_{\max} \leq 0
  \end{split}
  \label{eq:gersh-row3}
\end{equation}

\paragraph{Row 4 ($V$):}
\begin{equation}
  \begin{split}
  m_V(-2\delta + 2\alpha_c + \kappa &+ \lambda_2 C'_{\max}) \\
  &+ m_I\beta_I C'_{\max} \leq 0
  \end{split}
  \label{eq:gersh-row4}
\end{equation}

\subsection*{Step 3: Metric Construction}

We seek $m_E, m_I, m_S, m_V > 0$ and $\alpha_c > 0$ satisfying
\eqref{eq:gersh-row1}--\eqref{eq:gersh-row4}. A natural choice is to
set the metric weights inversely proportional to the coupling they
receive:
\begin{equation}
  m_E = 1, \quad m_I = \frac{\alpha}{2\gamma_I}, \quad
  m_S = 1, \quad m_V = \frac{\alpha}{2\delta}
\end{equation}

\subsection*{Step 4: Numerical Verification}

Production parameters: $\alpha = 0.42$, $\beta_E = 0.10$,
$\beta_I = 0.30$, $\gamma_I = 0.25$, $k = 0.10$, $\mu = 0.80$,
$\kappa = 0.30$, $\delta = 0.40$, $\lambda_2 = 0.05$,
$C'_{\max} = 0.5$.

\paragraph{Initial attempt.}
The natural metric $m_E = 1$, $m_I = 0.84$, $m_S = 1$,
$m_V = 0.525$ fails Row~2 ($+0.21 + 1.68\alpha_c > 0$).
Reducing $m_I = 0.2$, $m_E = 0.1$ makes Row~2 the binding
constraint:
\[
  -0.008 + 0.4\alpha_c \leq 0
  \;\implies\; \alpha_c < 0.02
\]
We set $\alpha_c = 0.019$ (strictly less) to ensure all
inequalities are strict.

\paragraph{Verification.}
With $M = \operatorname{diag}(0.1, 0.2, 1.0, 0.08)$ and
$\alpha_c = 0.019$:

\smallskip
\begin{center}
\footnotesize
\begin{tabular}{@{}lrrl@{}}
\toprule
\textbf{Row} & \textbf{LHS} & \textbf{$< 0$?} & \\
\midrule
1\,($E$) & $-0.006$ & yes & \checkmark \\
2\,($I$) & $-0.0004$ & yes & \checkmark \\
3\,($S$) & $-1.529$ & yes & \checkmark \\
4\,($V$) & $-0.005$ & yes & \checkmark \\
\bottomrule
\end{tabular}
\end{center}

\smallskip
Row~4 initially failed with $m_V = 0.05$ ($+0.008$); adjusting
to $m_V = 0.08$ resolves it ($-0.005$) while Row~1 remains
satisfied ($-0.006$). Row~2 now has strict margin ($-0.0004$).

\medskip
\textbf{Result:} With $M = \operatorname{diag}(0.1, 0.2, 1.0,
0.08)$, the Gershgorin analysis certifies $\alpha_c = 0.019$.
This is conservative---the eigenvalue-based bound is
$\alpha_c \approx 0.13$ and the spectral abscissa of~$J$ at
equilibrium is $-0.15$
(Remark~\ref{rem:production-params}). The key result is
$\alpha_c > 0$, guaranteeing global exponential
convergence. \qed


\section{Saturation Diagnostics}
\label{app:saturation}

The saturation margin $m_{\text{sat}} = 1 - 4A/\gamma_I$ was computed
for all 75 active agents using their current EISV state. Under the
logistic mode with production parameters ($\gamma_I = 0.25$,
$k = 0.10$, $\beta_I = 0.30$):

\begin{center}
\small
\begin{tabular}{@{}lrr@{}}
\toprule
\textbf{Regime} & \textbf{Mean $m_{\text{sat}}$} & \textbf{\% Sat.} \\
\midrule
Convergence ($n=24$) & $-1.45$ & 100\% \\
Exploration ($n=22$) & $-1.18$ & 100\% \\
Divergence ($n=29$) & $-1.10$ & 100\% \\
\midrule
All active ($n=75$) & $-1.23$ & 100\% \\
\bottomrule
\end{tabular}
\end{center}

All active agents operate with negative saturation margin under the
logistic mode, meaning the forcing term $A = -kS + \beta_I C(V)$
exceeds $\gamma_I/4 = 0.0625$ for all agents. The system currently
relies on state clamping at $I = 1$ rather than intrinsic dynamics to
prevent boundary escape. This motivates the linear correction
(\Cref{sec:saturation}): under the linear mode with $\gamma_I = 0.169$,
$I^* = A / \gamma_I \approx 0.80$, which is well within the interior
of $[0,1]$.

\textbf{Data provenance caveat.} These statistics reflect agents
accumulated since initial deployment (December 2025), during which
dynamics parameters, governor thresholds, and the I-channel mode itself
were iteratively refined. The 19 boundary-saturated agents predate the
linear mode rollout. We report the snapshot as-is rather than filtering
to a ``clean'' parameter regime, since the saturation criterion
(\Cref{prop:saturation}) depends only on the relationship $A > \gamma_I/4$,
which holds regardless of the agent's configuration history.


\end{document}
